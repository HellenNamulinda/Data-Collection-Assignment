\documentclass[10pt,letterpaper]{article}
\begin{document}
\title{VEHICLE THEFT:\\
A CASE STUDY ON RECOVERY OF LOST VEHICLES IN UGANDA.}
\author{by Namulinda Hellen  \\ 216000850 \\  16/U/900}
\maketitle
\section{INTRODUCTION }
 This research examines the issues of how lost vehicles are traced for and recovered by the owners.  Vehicle theft is one of the most commonly reported crimes in many Ugandan urban centres.  The Insurance Corporation of Uganda has recently indicated that, on an average day, 15 vehicles are stolen in Uganda. For the most part, research on Vehicle theft has focused on the motivations behind the act, the various methods employed by thieves, and, more recently, the profile and criminal development of Vehicle thieves. Given this, this report will examine the phenomenon of Vehicle theft in Uganda by examining the characteristics of stolen vehicles. Specifically, this report will include how to enhance recovery of vehicles from within the urban areas of Uganda. 
\section{Problem statement }
Vehicle theft is one of the most commonly reported crimes in many Ugandan urban centres. In general, research has identified three forms of Vehicle theft: (1) recreational theft for the purpose of transportation; and (3) theft for profit). Vehicle theft for recreational purposes typically involves joyriding or stealing a vehicle for fun with no real destination or other motive in mind. This category of Vehicle theft is often engaged in by youth looking to obtain status among their peers or who experience some psychological/physical thrill associated with engaging in or participating in a Vehicle theft. Vehicle theft for transportation involves stealing a vehicle for a single-trip, transportation associated with or facilitating the commission. Although most stolen vehicles are recovered by the police, there is little information regarding the nature of vehicle recovery.
\section{Main objective}
 The main objective of this research is to address the theft and recovery of vehicles. Information collected included the date of theft, location of theft, location of recovery, date of recovery, and damage to the vehicle.  
\section{Specific Objectives: }
1.	To determine how long it takes to trace for a lost vehicle.\\
2.	 To determine if the vehicle is lost, which methods are used to trace for a vehicle.\\ 
3.	 To find out the costs associated with recovering a vehicle.

\section{  RESEARCH QUESTIONS }
 The following research questions guided the study.\\
How does the police trace for the lost vehicles?\\
How long does it take to recover a vehicle?\\
Where do vehicles always get lost form?

\section{SCOPE OF THE STUDY}
 The study seeks to find out how lost vehicles are recovered in urban areas. The study will focus itself on urban areas of Uganda in the selected districts.
\section{ LIMITATIONS OF THE STUDY}
 The study is limited by time and financial resources and as result the research will have to source for more financial resources and use alternative means. 
\section{CONCLUSION}
Research on Vehicle theft has primarily focused on theft hot spots and the methods and motivations of stealing vehicles. Little attention has been paid to the nature of recovered vehicles. Examining the nature of recovered vehicles may provide policy makers with some insight into the causes of Vehicle theft and why vehicles are being abandoned within their jurisdictions.



\end{document}